\renewcommand{\headrulewidth}{0pt}
\fancyhead[R]{}

\begin{titlepage}
\newpage
\pagestyle{fancy}      
\lhead{}  
\chead{}     
\rhead{}     
    
\renewcommand{\headrulewidth}{0.5pt}

\begin{center}\huge{\textbf{R�sum�}} \\ \end{center} 

Au cours de cette d�cennie, l'industrie de la Technologies de l'information et de la communication (TIC) a connu une progression fulgurante suite � l'apparition des smartphones. Voulant exploiter ces technologies pour aider l'�tudiant � mieux g�rer son temps et � �tre plus productif, nous avons r�alis� une application Android qui permet � l'utilisateur de g�rer plusieurs calendrier th�matiques, d'y ajouter des �v�nements et la possibilit� de recevoir des alertes. Pour mener � terme ce projet, nous avons d'abord analys� les besoins, propos� une solution que nous avons mod�lis� gr�ce � UML, et impl�ment� en utilisant Android Studio, Java, XML, SQLite, Git, GitHub, et d'autres librairies open source. \\

\emph{\textbf{Mots cl�s:} Gestion du temps, Application mobile, Android, Room, SQLite.}
 

\paragraph{}
\paragraph{}

\begin{center}\huge{\textbf{Abstract}} \\ \end{center}

%During the last decade, the industry of information and technology (IT) has experienced a huge rise. In the hope of helping students, we exploited these technologies to allow them better management of their time and increased productivity ; we created an Android application that allows the user to manage several thematic calendars, to add events and and to recieve allerts for said events . In order to complete this project, we first analyzed the needs, proposed a solution that we modeled using UML, and then implemented using Android Studio, Java, XML, SQLite, Git, GitHub, and other open libraries.  %

During this decade,Information technology (IT) industry has experienced a huge rise following the appearance of smartphones. Wanting to exploit these technologies to help the student better manage his time and be more productive, we have created an Android application that allows the user to manage several thematic calendar, to add events and the possibility of receiving alerts. To complete this project, we first analyzed the needs, proposed a solution that we modeled using UML, and implemented using Android Studio, Java, XML, SQLite, Git, GitHub, and other open source  libraries.//

\emph{\textbf{Keywords:} Time scheduling, Mobile application, Android, Room, SQLite.}
 

\end{titlepage}