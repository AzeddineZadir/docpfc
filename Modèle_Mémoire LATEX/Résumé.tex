\renewcommand{\headrulewidth}{0pt}
\fancyhead[R]{}

\begin{titlepage}
\newpage
\pagestyle{fancy}      
\lhead{}  
\chead{}     
\rhead{}     
    
\renewcommand{\headrulewidth}{0.5pt}

\begin{center}\huge{\textbf{R�sum�}} \\ \end{center} 

Au cours de cette d�cennie, l'industrie de la t�l�phonie mobile a connu une progression fulgurante suite � l'apparition des smartphones. Voulant exploiter ces technologies pour aider l'�tudiant � mieux g�rer son temps et � �tre plus productif, nous avons r�alis� une application Android qui permet � l'utilisateur de g�rer plusieurs calendrier th�matiques, d'y ajouter des �v�nements et la possibilit� de recevoir des alertes. Pour mener � terme ce projet, nous avons d'abord analys� les besoins, propos� une solution que nous avons mod�lis� gr�ce � UML, et impl�ment� en utilisant Android Studio, Java, XML, SQLite, Git, GitHub, et d'autres librairies open source. \\

\emph{\textbf{Mots cl�s:} Gestion du temps, Application mobile, Android, Room, SQLite.}
 

\paragraph{}
\paragraph{}

\begin{center}\huge{\textbf{Abstract}} \\ \end{center}

At the end of this decade, the mobile phone industry has experienced a huge rise, following the appearance of smartphones. Wanting to help the student to better organize their schedule and to be more productive. We realized this project. Implementing the modeled solution during design follows needs analysis. Using UML as the modeling language, Android Studio, Java, XML, SQLite, Git, GitHub, and Room.\\

\emph{\textbf{Keywords:} Time scheduling, Mobile application, Android, Room, SQLite.}
 

\end{titlepage}