\chapter{Impl�mentation}
\fancyhead[R]{\textit{Impl�mentation}}
\renewcommand{\headrulewidth}{1pt}


\section{Introduction}

 Ce dernier chapitre est consacr� � la partie pratique de la r�alisation de notre projet. Dans un premier temps, nous allons �num�rer les diff�rents outils de d�veloppement qui nous ont permis de mener � bien notre application mobile. Ensuite, nous allons pr�sent� les diff�rents langage de programmation utilis�s, les librairies et enfin les diff�rentes interfaces de notre application.  


\section{Environnement de d�veloppement}


\subsection{Android Studio}

Android Studio est un environnement de d�veloppement int�gr�(EDI) permettant de d�velopper des application sous Android. D�velopp� par Google, il se base sur l'EDI IntelliJ de JetBrains. Il offre les outils n�cessaires pour d�velopper des applications mobiles natives destin�es � Android. Ainsi, il permet d'�diter des fichiers Java/Kotlin  

\subsection{Git}

Git est un logiciel libre de gestion de versions, sous licence publique g�n�rale GNU 2. 

\subsection{GitHub}
GitHub est un service web de gestion et d'h�bergement de projet de d�veloppement logiciel utilisant le logiciel Git.  

\subsection{SDK de Android}

Le SDK (Software Developpement Kit) de Android est un ensemble d'outils de d�veloppement essentiel au d�veloppement d'application mobile sous Android, il inclut ainsi de diff�rents outils tel que un d�bogueur, de la documentation, un �mulateur bas� sur QEMU, un ensemble de biblioth�ques logicielles.  

\subsection{JDK}

\subsection{Langage de programmation}

\subsection{Java}
\subsection{XML}
\subsection{SQL}
La figure \ref{fig1} ...

\begin{figure}[h!]
	\centering
		\includegraphics[width=8cm]{images/MySQL.png}
	\caption{Titre de la figure.}
	\label{fig1}
\end{figure}


\subsection{Titre de la deuxi�me sous-section}

Texte ...


%%%% Ajout d'un tableau %%%%
\begin{table}[!h]
		\small
		\centering
		\footnotesize{
			\begin{tabular}{|p{4cm}|p{4cm}|p{4cm}|} %%% La taille des trois colonnes est �gale � 4cm %%%
				\hline
				\rowcolor{lightgray} \textbf{Colonne 1} & \textbf{Colonne 2} & \textbf{Colonne 3} \\
				\hline
				Ligne 1 Colonne 1 & Ligne 1 Colonne 2 & Ligne 1 Colonne 3 \\
				\hline
				Ligne 2 Colonne 1	& etc. 							& etc. \\
				\hline
			\end{tabular}}
		\caption{Titre du tableau.}	
\end{table}
	

\section{Conclusion}

Texte ...